%\begingroup
%\let\cleardoublepage\clearpage


% English abstract
\cleardoublepage
\chapter*{Abstract}
%\markboth{Abstract}{Abstract}
\addcontentsline{toc}{chapter}{Abstract} % adds an entry to the table of 
% put your text here
Many algorithms have been proposed to solve the problem of matching 
feature points in two or more images using geometric assumptions to 
increase the robustness of the matching. However, very little work 
addresses the cases where these assumptions might not hold. In 
particular, few methods address the problem of reliable matching in 
cases where it is unknown whether two images have any corresponding 
areas or objects in the first place.  \\

We propose two algorithms for matching feature points without the use of 
geometric constraints. The first relies on the idea that any match 
between two images should be better than all possible matches within a 
single image. The second algorithm extends this idea by using the 
community structure in the similarity graph of feature points to find 
reliable correspondences. To evaluate the algorithms experimentally, we 
introduce a simple method to generate a large number of test cases based 
on a set of image pairs with viewpoint changes. Our results show that 
the proposed algorithm is generally superior to traditional approaches 
in finding correct correspondences.
