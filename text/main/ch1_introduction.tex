\addcontentsline{toc}{chapter}{Introduction}
\chapter*{Introduction}
%Matching feature points is one of the problems in computer vision that 
%if solved well will provide a useful tool for improving methods on a 
%host of interesting problems. If we can match feature points reliably 
%we can recognize objects, make panoramas, recognize duplicate photos, 
%guide robots and airplanes and a host of other interesting applications 
%\footnote{For a few examples of some of these applications, check out: 
%\cite{se2002global} (robot guidance), \cite{brown2007automatic} 
%(panorama construction), \cite{lowe2004sift} (has parts on object 
%recognition), \cite{chu2010consumer} (duplicate detection)}.  For all 
%these problems better feature point matching will most likely yield 
%better solutions.

My original goal when I started working on this thesis was not to 
improve feature point matching. I started out working on face 
clustering, but via a series of detours I ended up circling the problem,
which I found alluring not just because it its importance, but also 
because of its inherent simplicity.  Out of two sets, finding pairs of 
similar elements seems like a problem that should have been solved long 
ago, yet because of the nature of feature points and the data they are 
used on, there still seems to be plenty of progress to be made.

As the outcome of my work I propose two algorithms, \emph{Mirror Match 
(MM)} and \emph{Mirror Match with Clustering (MMC)}. Both algorithms are 
designed to tackle the problem of reliably matching feature points under 
circumstances where we cannot assume two images necessarily have any 
correspondences.  The goal of this thesis will be to describe these 
algorithms and convincingly demonstrate that they outperform existing 
approaches.  While doing this I hope to provide sufficient background 
for a technically-minded reader without any necessary experience in 
computer vision to follow the motivation behind \emph{MM} and \emph{MMC} 
and enable him or her to understand both algorithms within a bigger 
scientific perspective.
 
The thesis is organized as follows: Chapter~\ref{C:Background} provides 
background and motivation for the development of \emph{MM} and 
\emph{MMC}. The chapter serves two purposes:  For those who might be 
reading this introduction wondering what a feature point really is and 
how it is matched, it provides a gentle introduction to the hows and 
whys of feature point matching. At the same time, the chapter is meant 
as an introduction to related work which is hopefully useful for those 
who might use this thesis as a starting point for finding other research 
on the same subject. Chapter~\ref{C:Algorithms} introduces in detail the 
algorithms proposed in this thesis. The reference algorithms used to 
benchmark \emph{MM} and \emph{MMC} are also introduced in detail.  
Chapter~\ref{C:Experiments} details both the setup and results of the 
experiments done to evaluate the proposed algorithms. 
Chapter~\ref{C:Applications} provides two examples of how the matching 
algorithms can improve on panorama stitching and near duplicate 
matching.

For the remainder of the thesis I will make use of the academic 
'\emph{we}' to underline the fact this thesis would not have been 
possible had it not been for the guidance of \emph{Stefan Winkler} and 
\emph{Sabine S\"usstrunk} as well as the support of \emph{Advanced 
Digital Science Center} in Singapore, which hosted me as an intern 
during the term in which I worked on this thesis. 
