\addcontentsline{toc}{chapter}{Introduction}
\chapter*{Introduction}
Matching feature points is one of the problems in computer vision that 
if solved well will provide a useful tool for improving methods on a 
host of interesting problems. If we can match feature points reliably we 
can recognize objects, make panoramas, recognize duplicate photos, guide 
robots and airplanes and a host of other interesting applications 
\footnote{For a few examples of some of these applications, check out: 
\cite{se2002global} (robot guidance), \cite{brown2007automatic} 
(panorama construction), \cite{lowe2004sift} (has parts on object 
recognition), \cite{chu2010consumer} (duplicate detection)}.  For all 
these problems better feature point matching will most likely yield 
better solutions.

My goal when I started working on this thesis was not to improve feature 
point matching, but via a series of detours I ended up circling the 
problem which attracted me, not just because it its importance, but also 
because of its inherent simplicity. Out of two sets, finding pairs of 
similar elements seems like a problem that should have been solved long 
ago, yet because of the nature of feature points and the data they are 
used on, there seems to be plenty of progress to be made.

As the outcome of my work  I propose two algorithms, \emph{Mirror Match 
(MM)} and \emph{Mirror Match with Clustering (MMC)}, both designed to 
improve feature point matching. The goal of this thesis will be to 
introduce these algorithms and convincingly demonstrate that they are 
useful. While doing this I hope to provide sufficient background for a 
technically minded reader without any necessary background in computer 
vision to follow the motivation behind \emph{MM} and \emph{MMC} as well 
as the place of both algorithms in a bigger scientific perspective.
 
The thesis is organized as follows: Chapter~\ref{C:background} provides 
background and motivation for the development of \emph{MM} and 
\emph{MMC}. The chapter serves two purposes.  For those that might be 
reading this introduction wondering what a feature point really is and 
how it is matched, it provides a gentle introduction to the why's and 
how's of feature point matching. At the same time the chapter is meant 
as an introduction to related work which is hopefully useful for those 
who might use this thesis as a starting point for finding other research 
on the same subject. Chapter~\ref{C:algorithms} will introduce in detail 
the algorithms proposed in this thesis. The reference algorithms used to 
benchmark \emph{MM} and \emph{MMC} are also introduced in detail.  
Chapter~\ref{C:experiments} details the experiments done to evaluate the 
proposed algorithms both in terms of how they were conducted as well as 
their results. Chapter~\ref{C:applications} provides two examples of how 
the matching algorithms can improve on panorama stitching and near 
duplicate matching.

For the remainder of the thesis I will make use of the academic 
'\emph{we}' to underline the fact this thesis had not been possible had 
it not been for the guidance of Stefan Winkler and Sabine S\"usstrunk as 
well as the support of Advanced Digital Science Center in Singapore that 
hosted me as an intern for the duration of my thesis work. 
