\documentclass{article}
\usepackage{graphicx}
\usepackage{hyperref}
\usepackage[stable]{footmisc}
\usepackage{caption}
\usepackage{adjustbox}
\usepackage{subcaption}
\usepackage{algpseudocode}
\usepackage{algorithm}
\usepackage{amsmath}
\usepackage{amssymb}

\begin{document}

\title{Clustering feature points}
\author{Jonas Arnfred}

\maketitle

\begin{abstract}
% TODO!
\end{abstract}

\section{Project description}

Traditionally the feature points of two images have been matched by 
comparing the feature points individually and finding the best matches 
based on their distance. This approach was originally introduced with 
the SIFT features \cite{lowe2004sift} and has been used since then in 
computer vision applications such as scene recognition\footnote{See for 
example chapter 4.1.3 of \cite[pp. 226]{szeliski2010}} and image 
alignment\footnote{An example is the gold standard algorithm in \cite[p.  
114]{multipleView}}. In larger scale approaches where an image is 
compared with thousands or millions of other images, the feature points 
are usually indexed in a dictionary of visual words instead. This 
approach makes it possible to quickly retrieve matches in a large set of 
potential candidates, but the performance is restricted by how 
discriminative the individual feature points are. To agument the 
descriminative power of the feature points, \cite{wu2009bundling} 
proposes grouping the features and use the geometrical relationships 
between members of a group when matching two feature points. Similarly 
\cite{das2008event} use geometrical information about the feature points 
to cluster them before doing scene matching resulting in better matches.

Both methods suffer from two shortcomings. Firstly we have that any 
defined group in one image might not correspond to a group in another 
image.  This means that we might end up in a situation where a good 
match between two images is scored low because most of it's geometric 
neighbors are in an adjecent group. Secondly if we intent to match the 
different groups as in \cite{wu2009bundling} we might end up matching 
two mismatching groups. To improve geometric matching I propose pooling 
the keypoints from two images into one set and partitioning them into 
groups together.  The advantages of this approach is that groups are 
consistent across images, and that keypoints from one image that don't 
have equivalents in the other image will simply be grouped in their own 
group which can easily be discarded during the matching. This is 
important for scene recognition where people might obscure details in 
the picture. In addition the proposed method generalizes to three or 
more images.  This makes it possible to easily assemble a set of 
partially overlapping pictures without any required knowledge of their 
position.  What's more, if one of the pictures in there is not part of 
the set, the features of the odd image out will not form clusters with 
the features of the other images and automatically be sorted out. 

\subsection{Graph Clustering}

Given a set of feature points from two images, $im_1$ and $im_2$: $F_k = 
\{k_i, k_j$ for $k_i \in im_1, k_j \in im_2\}$ as well as a matching 
function $M(k_i, k_j) \rightarrow \mathbb{R}$ that takes two feature 
points in an image and return their matching score, we can define a 
matrix $A$ where each element $A_ij = M(k_i, k_j)$. A can be interpreted 
as the \emph{adjecency matrix} of the fully connected graph where each 
vertex corresponds to a keypoint and the edge between two vertices has a 
weight equal to the distance between the two corresponding keypoints.

This representation reduces the problem of partitioning the keypoints 
into groups reduces the problem to that of graph clustering or community 
structure depending on the context. In the litterature there are various 
ways of clustering a graph according to different measures of what 
constitutes an optimal partitioning. Traditionally the most used 
clustering algorithms have been K-means and spectral clustering, but in 
recent years a host of new algorithms have been proposed based on both 
Newman's concept of graph modularity\footnote{Introduced in 
\cite{girvan2002}, discussed in \cite{brandes2007} and used in 
\cite{blondel2008} as well as others} as well as information theoretical 
measures\footnote{See for example \cite{rosvall2008}} and the Potts spin 
model from physics\footnote{Used in \cite{ronhovde2009}} just to mention 
a few approaches. Many of the new algorithms differ from K-means 
clustering and Spectral clustering in that they don't require the number 
of expected to clusters to be specified beforehand\footnote{Among the 
aforementioned methods, this is true for \cite{blondel2008} and 
\cite{rosvall2008}}.  Furthermore, on tests done using randomly 
generated graphs with a known partitioning \cite{blondel2008}, 
\cite{rosvall2008} and \cite{ronhovde2009} perform markedly better than 
spectral clustering and K-means\cite{lancichinetti2009}.

The performance of clustering algorithms is a complicated issue, since 
an optimal clustering given the same graph can vary depending on the 
application. Spectral clustering for example will usually return a 
partitioning where each partition is roughly equal in size\footnote{As 
mentioned in \cite{von2007}} while the Louvain 
clustering\cite{blondel2008} might return partitions of very uneven 
size, even if the modularity measurement has been shown to penalize very 
small clusters\cite{brandes2007}. Both behaviours can be beneficial 
depending on the application, but when clustering feature points, 
maintaining clusters of an even size usually means that some clusters 
will be '\emph{catch-all}' clusters where the feature points that don't 
fit anywhere else are grouped together. The necessity of specifying the 
amount of partitions in for example Spectral clustering or Pott's model 
clustering further exacerbates the issue since smaller partitions are 
then combined into one to achieve the right amount of partitions.

Based on these observations I've decided to use Louvain clustering for 
grouping feature points since it's fast, performs well, doesn't require 
parameters and doesn't emphasize partitions of equal sizes.

\subsection{Clustering feature points using Louvain clustering}

\begin{algorithm}
\caption{Louvain-Match Algorithm}
\label{alg-simple}
\begin{algorithmic}
\Require $images$ : set of images, $n$ : int $\in \mathbb{N}$
\State $matches\gets \varnothing$
\ForAll{$I_i \in images$} \Comment Gather features of all images
	\State $f_i\gets getFeatures(I_i)$
\EndFor
\State $A\gets getAdjecencyMatrix(f_1, f_2,\; \ldots \;, f_n)$
\State $A_{norm}\gets 1 - normalize(A)$
\Comment Shorter distance is higher weight
\State $A_{pruned}\gets pruneEdges(A_{norm},\alpha)$
\State $P\gets cluster(A_{pruned})$ 
\Comment P is a set of partitions
\ForAll{$p \in P$} \Comment p is a set of feature points
	\State $trait\gets \left\{p \cap f_i \mid p \cap f_i \neq 
\varnothing, f_i \in \left\{f_1, f_2,\; \ldots \;, f_n\right\}\right\}$
	\If{$\left\vert trait \right\vert > 1$}
		\State $matches\gets getPositions(trait)$
		\Comment Position of trait in each image
	\EndIf
\EndFor \\
\Return matches
\end{algorithmic}
\end{algorithm}

Algorithm \ref{alg-simple} shows the basic steps taken to find matches 
in two or more images. This algorithm is a simplified version of the 
proposed method completely agnostic of the geometry of the feature 
points in the picture. To better explain the algorithm here is an 
overview over the inputs and functions mentioned and how they work:
\begin{itemize}
\item[]{\emph{images}: The images we are matching}
\item[]{\emph{$n$}: The number of weights per row in the adjecency 
matrix}
\item[]{\emph{getFeatures}: Returns a set of feature points given an 
image. The type of feature point is not theoretically important, but in 
practice the BRIEF\cite{calonder2010brief} feature point detector and 
descriptor has worked well} 
% TODO: include a comparison of SURF, SIFT, etc
\item[]{\emph{getAdjecencyMatrix}: Based on the distance measure used to 
compare the similarity of the feature points a symmetric adjecency 
matrix is returned}
\item[]{\emph{pruneEdges}: The result of the Louvain clustering depends 
on the density of the graph. To ensure an optimal number of partitions 
we only keep the $n$ highest weights per row and column in the adjecency 
matrix, making sure it's still symmetric afterwards. In practice setting 
$n$ to around four or five usually yields good results}
\item[]{\emph{cluster}: Cluster the graph corresponding to the adjecency 
matrix given using the Louvain algorithm in \cite{blondel2008}}
\item[]{\emph{getPositions}: Given a trait consisting of features in 
several different images, find the positions of traits in each image and 
return them. This could either be done by taking a mean of the feature 
points in an image, or by returning the position of the highest matching 
feature points within the trait. In practice ... TODO}
\end{itemize}

\begin{algorithm}
\caption{Louvain-Match Algorithm With Geometry}
\label{alg-geometry}
\begin{algorithmic}
\Require $images$ : set of images, $\alpha$ : float $\in \left[0, 
1\right]$, $\beta$ : float $\in \left[0, 1\right]$
\State $matches\gets \varnothing$
\ForAll{$I_i \in images$} \Comment Gather features of all images
	\State $f_i\gets getFeatures(I_i)$
\EndFor
\State $A\gets getAdjecencyMatrix(f_1, f_2,\; \ldots \;, f_n)$
\State $B\gets getGeomDistanceMatrix(f_1, f_2,\; \ldots \;, f_n)$
\State $M\gets getInterImageMask(f_1, f_2,\; \ldots \;, f_n)$
\State $B_{norm}\gets 1 - normalize(G)$ \State $A_{norm}\gets 1 - 
normalize(A)$
\State $A_{geom}\gets A_{norm}\left[ \neg M\right] + \beta \cdot 
B_{norm}\left[M\right]$
\State $A_{pruned}\gets pruneEdges(A_{geom},\alpha)$
\State $P\gets cluster(A_{pruned})$ 
\ForAll{$p \in P$}
	\State $trait\gets \left\{p \cap f_i \mid p \cap f_i \neq 
\varnothing, f_i \in \left\{f_1, f_2,\; \ldots \;, f_n\right\}\right\}$
	\If{$\left\vert trait \right\vert > 1$}
		\State $matches\gets getPositions(trait)$
		\Comment Position of trait in each image
	\EndIf
\EndFor \\
\Return matches
\end{algorithmic}
\end{algorithm}

While clustered feature points are often geometrically close together, 
there might be cases where several points in the same image are similar 
enough to all be included in one cluster despite their geometrical 
distance.  To solve this problem we can either modify the clustering 
algorithm to take the geometric distances into account like 
\cite{das2008event}.  Alternatively we can modify the weight matrix to 
penalize putting distanced feature points in the same cluster. However 
we only know the geometrical distance between two feature points as long 
as both belong to the same image. One way to approach this problem is to 
replace all the weights of edges between two feature points within the 
same image with their distance. For the case of two images the original 
adjecency matrix is defined as follows:
%
$$A=
\begin{bmatrix}
w_{1,1}   & w_{1,2}   & \ldots & w_{1,n-1}   & w_{1,n}     \\
w_{2,1}   & w_{2,2}   & \ldots & w_{2,n-1}   & w_{2,n}     \\
\vdots    & \vdots    & \ddots & \vdots      & \vdots      \\
w_{n-1,1} & w_{n-1,2} & \ldots & w_{n-1,n-1} & w_{n-1,n}   \\
w_{n,1}   & w_{n,2}   & \ldots & w_{n,n-1}   & w_{n,n}
\end{bmatrix}=
\begin{bmatrix}
A_{im1} & S       \\
T       & A_{im2} \\
\end{bmatrix}
$$
%
In a similar manner we can define an adjecency matrix based on the 
geometrical distance. Note how for example $d_{1,n}$ doesn't contain any 
sensible information.
%
$$B=
\begin{bmatrix}
d_{1,1}   & d_{1,2}   & \ldots & d_{1,n-1}   & d_{1,n}     \\
d_{2,1}   & d_{2,2}   & \ldots & d_{2,n-1}   & d_{2,n}     \\
\vdots    & \vdots    & \ddots & \vdots      & \vdots      \\
d_{n-1,1} & d_{n-1,2} & \ldots & d_{n-1,n-1} & d_{n-1,n}   \\
d_{n,1}   & d_{n,2}   & \ldots & d_{n,n-1}   & d_{n,n}
\end{bmatrix}=
\begin{bmatrix}
B_{im1} & R       \\
Q       & B_{im2} \\
\end{bmatrix}
$$
%
Given these two and a weight factor $\beta$ we can construct a new 
weight matrix as follows:
%
$$C=
\begin{bmatrix}
\beta \cdot B_{im1} & S       \\
T       & \beta \cdot B_{im2} \\
\end{bmatrix}
$$
%
Since we prune the matrix afterwards, it's important to choose a weight 
factor that doesn't lead to all edges being either intra image edges or 
intra image edges, since the first would create no clusters across 
images and the latter would mean that the geometrical information would 
be discarded and the bipartite graph constituting of edges between 
feature points in the two images would be used instead.

\bibliographystyle{abbrv}
\bibliography{bibliography}
\end{document}

